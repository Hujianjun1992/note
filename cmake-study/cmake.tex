\documentclass{beamer}
\usetheme{Warsaw}
\usecolortheme{lily}

\usepackage{CJKutf8}

\newcommand{\quotes}[1]{"#1"}

\begin{document}

\begin{CJK}{UTF8}{gkai}

  \title{Cmake}
  \subtitle{笔记}
  \author{H-X-B \LaTeX{} }
  \institute[HIT]
  {
    哈尔滨工业大学~航天学院\\
    \medskip
    \textit{292554331@qq.com}\\
  }
  \date{\today}

  \frame{\titlepage}

  \begin{frame}\frametitle{模板}
    cmake\_minimum\_required(VERSION 2.8.3 FATAL\_ERROR)\\
    project(Cmake)\\
    set(SRC\_LIST main.cpp)\\
    message(STATUS "" \${PROJECT\_BINARY\_DIR})\\
    message(STATUS "" \${PROJECT\_SOURCE\_DIR})\\
    add\_executable(cmake1 \${SRC\_LIST})\\
    make VERBOSE=1 看到 make 构建的详细过程
  \end{frame}

  \begin{frame}\frametitle{}
    SET(VAR [VALUE] [CACHE TYPE DOCSTRING [FORCE]])\\
    MESSAGE([SEND\_ERROR | STATUS | FATAL\_ERROR] "message to
    display"...)\\
    SEND\_ERROR 产生错误,生成过程被跳过。\\
    SATUS 输出前缀为—的信息。\\
    FATAL\_ERROR,立即终止所有 cmake 过程.
  \end{frame}

  \begin{frame}\frametitle{}
    ADD\_SUBDIRECTORY(source\_dir [binary\_dir] [EXCLUDE\_FROM\_ALL])\\
    这个指令用于向当前工程添加存放源文件的子目录,并可以指定中间二进制和目标二进制存
    放的位置。EXCLUDE\_FROM\_ALL 参数的含义是将这个目录从编译过程中排
    除,\\
    ADD\_SUBDIRECTORY(src bin)\\
    上面的例子定义了将 src 子目录加入工程,并指定编译输出(包含编译中间结果)路径为
    bin 目录。如果不进行 bin 目录的指定,那么编译结果(包括中间结果)都将存放在
    build/src 目录(这个目录跟原有的 src 目录对应),指定 bin 目录后,相当于在编译时
    将 src 重命名为 bin,所有的中间结果和目标二进制都将存放在 bin 目录。
  \end{frame}

  \begin{frame}\frametitle{}
    SET(EXECUTABLE\_OUTPUT\_PATH \${PROJECT\_BINARY\_DIR}/bin)\\
    SET(LIBRARY\_OUTPUT\_PATH \${PROJECT\_BINARY\_DIR}/lib)\\
    在哪里 ADD\_EXECUTABLE 或 ADD\_LIBRARY,如果需要改变目标存放路径,就在哪里加入上述的定义
  \end{frame}

  \begin{frame}\frametitle{目标文件}
%    INSTALL(TARGETS targets...\\
%    [[ARCHIVE|LIBRARY|RUNTIME]\\
%    [DESTINATION <dir>]\\
%    [PERMISSIONS permissions...]\\
%    [CONFIGURATIONS\\
%    [Debug|Release|...]]\\
%    [COMPONENT <component>]\\
%    [OPTIONAL]\\
%    ] [...])\\
    目标类型也就相对应的有三种,ARCHIVE 特指静态库,LIBRARY 特指动态库,RUNTIME
    特指可执行目标二进制。\\
    DESTINATION 定义了安装的路径,如果路径以/开头,那么指的是绝对路径,这时候
    CMAKE\_INSTALL\_PREFIX 其实就无效了。如果你希望使用 CMAKE\_INSTALL\_PREFIX 来
    定义安装路径,就要写成相对路径,即不要以/开头,那么安装后的路径就是
    \${CMAKE\_INSTALL\_PREFIX}/<DESTINATION 定义的路径>
  \end{frame}

  \begin{frame}\frametitle{目标文件}
    INSTALL(TARGETS myrun mylib mystaticlib\\
    RUNTIME DESTINATION bin\\
    LIBRARY DESTINATION lib\\
    ARCHIVE DESTINATION libstatic\\
    )\\
    可执行二进制myrun安装到\${CMAKE\_INSTALL\_PREFIX}/bin 目录\\
    动态库libmylib安装到\${CMAKE\_INSTALL\_PREFIX}/lib 目录\\
    静态库libmystaticlib安装到\${CMAKE\_INSTALL\_PREFIX}/libstatic 目录\\
  \end{frame}

  \begin{frame}\frametitle{普通文件}
    % INSTALL(FILES files... DESTINATION <dir>
    % [PERMISSIONS permissions...]
    % [CONFIGURATIONS [Debug|Release|...]]
    % [COMPONENT <component>]
    % [RENAME <name>] [OPTIONAL])
  \end{frame}

  \begin{frame}\frametitle{目录}
    % INSTALL(DIRECTORY dirs... DESTINATION <dir>
    % [FILE_PERMISSIONS permissions...]
    % [DIRECTORY_PERMISSIONS permissions...]
    % [USE_SOURCE_PERMISSIONS]
    % [CONFIGURATIONS [Debug|Release|...]]
    % [COMPONENT <component>]
    % [[PATTERN <pattern> | REGEX <regex>]
    % [EXCLUDE] [PERMISSIONS permissions...]] [...])
  \end{frame}

  \begin{frame}\frametitle{Example}
    INSTALL(DIRECTORY icons scripts/ DESTINATION share/myprojPATTERN \quotes{CVS} EXCLUDE\\
    PATTERN \quotes{scripts/*}\\
    PERMISSIONS OWNER\_EXECUTE OWNER\_WRITE OWNER\_READ\\
    GROUP\_EXECUTE GROUP\_READ)\\
    将 icons 目录安装到 <prefix>/share/myproj,将scripts/中的内容安装到
    <prefix>/share/myproj 不包含目录名为 CVS 的目录,对于 scripts/* 文件指定权限为 OWNER\_EXECUTE
    OWNER\_WRITE OWNER\_READ GROUP\_EXECUTE GROUP\_READ.
  \end{frame}

  \begin{frame}\frametitle{}
    安装 COPYRIGHT/README: INSTALL(FILES COPYRIGHT README DESTINATION
    share/doc/cmake/t2)\\
    安装 runhello.sh(脚本):INSTALL(PROGRAMS runhello.sh DESTINATION
    bin)\\
    安装 doc 中的 hello.txt: INSTALL(DIRECTORY doc/ DESTINATION
    share/doc/cmake/t2)\\
  \end{frame}

  \begin{frame}\frametitle{命令}
    cmake -DCMAKE\_INSTALL\_PREFIX=/tmp/t2/usr .. \\
    make \\
    make install\\
  \end{frame}

  \begin{frame}\frametitle{静态库与动态库}
    SET(LIBHELLO\_SRC hello.c)\\
    ADD\_LIBRARY(hello SHARED \${LIBHELLO\_SRC})\\
    SET(LIBRARY\_OUTPUT\_PATH <路径>)来指定一个新的位置\\
  \end{frame}

   \begin{frame}\frametitle{ADD\_LIBRARY}
    % ADD_LIBRARY(libname
    % [SHARED|STATIC|MODULE]
    % [EXCLUDE\_FROM\_ALL]
    % source1 source2 ... sourceN)
     SHARED,动态库\\
     STATIC,静态库\\
   \end{frame}

  \begin{frame}\frametitle{添加静态库}
    ADD\_LIBRARY(hello\_static STATIC \${LIBHELLO\_SRC}) 就可以构建一个
    libhello\_static.a 的静态库了。\\
    SET\_TARGET\_PROPERTIES(hello\_static PROPERTIES OUTPUT\_NAME
    \quotes{hello})将libhello\_static.a改名为libhello.a\\
  \end{frame}

  \begin{frame}\frametitle{安装}

  \end{frame}

  \begin{frame}\frametitle{命令}
    SET\_TARGET\_PROPERTIES,其基本语法是:
    SET\_TARGET\_PROPERTIES(target1 target2 ...\\
    PROPERTIES prop1 value1\\
    prop2 value2 ...)\\
    GET\_TARGET\_PROPERTY(VAR target property)\\
    Example:GET\_TARGET\_PROPERTY(OUTPUT\_VALUE hello\_static
    OUTPUT\_NAME)\\
    MESSAGE(STATUS \quotes{This is the hello\_static OUTPUT\_NAME:}\${OUTPUT\_VALUE})
  \end{frame}

  \begin{frame}\frametitle{命令}
    SET\_TARGET\_PROPERTIES(hello PROPERTIES CLEAN\_DIRECT\_OUTPUT 1)\\
    SET\_TARGET\_PROPERTIES(hello\_static PROPERTIES
    CLEAN\_DIRECT\_OUTPUT 1)\\
    没有用  有可能版本的问题
  \end{frame}

  \begin{frame}\frametitle{动态库版本号}
    SET\_TARGET\_PROPERTIES(hello PROPERTIES VERSION 1.2 SOVERSION
1)\\
    VERSION 指代动态库版本,SOVERSION 指代 API 版本。\\
  \end{frame}

  \begin{frame}\frametitle{安装库及头文件}
    INSTALL(TARGETS hello hello\_static\\
    LIBRARY DESTINATION lib\\
    ARCHIVE DESTINATION lib)\\
    INSTALL(FILES hello.h DESTINATION include/hello)\\
  \end{frame}

  \begin{frame}\frametitle{使用共享库和头文件}
    方法一:include<hello/hello.h>\\
    方法二:INCLUDE\_DIRECTORIES(/usr/include/hello)\\
    方法三: export CMAKE\_INCLUDE\_PATH=/usr/include/hello\\
    FIND\_PATH(myHeader hello.h)\\
    IF(myHeader)\\
    INCLUDE\_DIRECTORIES(\${myHeader})\\
    ENDIF(myHeader)\\
    TARGET\_LINK\_LIBRARIES(main hello)
  \end{frame}

  \begin{frame}\frametitle{命令}
    LINK\_DIRECTORIES(directory1 directory2 ...)\\
    添加非标准的共享库搜索路径,比如,在工程内部同时存在共享库和可执行二
    进制,在编译时就需要指定一下这些共享库的路径。\\
    TARGET\_LINK\_LIBRARIES(target library1\\
    <debug | optimized> library2\\
    ...)\\
  \end{frame}

  \begin{frame}\frametitle{指令}
    使用\$ENV{NAME}指令就可以调用系统的环境变量\\
    测试:add\_test(mytest \${PROJECT\_BINARY\_DIR}/src/cmake)\\
    enable\_testing()\\
    make test\\
  \end{frame}

  \begin{frame}\frametitle{}
    if(WIN32)\\
    message(STATUS \quotes{This is windows})\\
    elseif(LINUX)\\
    message(STATUS \quotes{This is Linux})\\
    elseif(UNIX)\\
    message(STATUS \quotes{This is UNIX})\\
    endif(WIN32)\\
  \end{frame}

  \begin{frame}\frametitle{}
    FOREACH(VAR RANGE 10)\\
    MESSAGE(\${VAR})\\
    输出:0 1 2 3 4 5 6 7 8 9 10
  \end{frame}

  \begin{frame}\frametitle{}
    FOREACH(A RANGE 5 15 3)\\
    MESSAGE(\${A})\\
    ENDFOREACH(A)\\
    输出:5 8 11 14
  \end{frame}

\end{CJK}

\end{document}