\documentclass{beamer}

\mode<presentation> {

% The Beamer class comes with a number of default slide themes
% which change the colors and layouts of slides. Below this is a list
% of all the themes, uncomment each in turn to see what they look like.

%\usetheme{default}
%\usetheme{AnnArbor}
%\usetheme{Antibes}
%\usetheme{Bergen}
\usetheme{Berkeley}
%\usetheme{Berlin}
%\usetheme{Boadilla}
%\usetheme{CambridgeUS}
%\usetheme{Copenhagen}
%\usetheme{Darmstadt}
%\usetheme{Dresden}
%\usetheme{Frankfurt}
%\usetheme{Goettingen}
%\usetheme{Hannover}
%\usetheme{Ilmenau}
%\usetheme{JuanLesPins}
%\usetheme{Luebeck}
%\usetheme{Madrid}
%\usetheme{Malmoe}
%\usetheme{Marburg}
%\usetheme{Montpellier}
%\usetheme{PaloAlto}
%\usetheme{Pittsburgh}
%\usetheme{Rochester}
%\usetheme{Singapore}
%\usetheme{Szeged}
%\usetheme{Warsaw}

% As well as themes, the Beamer class has a number of color themes
% for any slide theme. Uncomment each of these in turn to see how it
% changes the colors of your current slide theme.

%\usecolortheme{albatross}
%\usecolortheme{beaver}
%\usecolortheme{beetle}
%\usecolortheme{crane}
\usecolortheme{dolphin}
%\usecolortheme{dove}
%\usecolortheme{fly}
%\usecolortheme{lily}
%\usecolortheme{orchid}
%\usecolortheme{rose}
%\usecolortheme{seagull}
%\usecolortheme{seahorse}
%\usecolortheme{whale}
%\usecolortheme{wolverine}

%\setbeamertemplate{footline} % To remove the footer line in all slides uncomment this line
%\setbeamertemplate{footline}[page number] % To replace the footer line in all slides with a simple slide count uncomment this line

%\setbeamertemplate{navigation symbols}{} % To remove the navigation symbols from the bottom of all slides uncomment this line
}

\usepackage{pgfpages}
\usepackage{latexsym,pifont,units,amsmath,amsfonts,amssymb,marvosym}

\newcommand{\quotes}[1]{"#1"}

\usepackage{graphicx} % Allows including images
\graphicspath{{./figs/}{../figs/}{./}{../}}

\usepackage{booktabs} % Allows the use of \toprule, \midrule and \bottomrule in tables
\usepackage[UTF8,noindent]{ctex}  %ctex
%\usepackage{fontspec}
\usepackage{color}
%\usepackage{xcolor}
%%-------------------------------------------------
\definecolor{keywordcolor}{rgb}{0.8,0.1,0.5}
\usepackage{listings}
\lstset{breaklines}%这条命令可以让LaTeX自动将长的代码行换行排版
\lstset{extendedchars=false}%这一条命令可以解决代码跨页时,章节标题,页眉等汉字不显示的问题
\lstset{language=C++, %用于设置语言为C++
         keywordstyle=\color{keywordcolor} \bfseries,%设置关键词
         identifierstyle=,
         basicstyle=\ttfamily,
         commentstyle=\color{blue} \textit,
         stringstyle=\ttfamily,
         showstringspaces=false,
         %frame=shadowbox, %边框
         captionpos=b
}

\hypersetup{CJKbookmarks=true} %解决section不能使用中文的问题

\usepackage{CJKutf8}

\begin{document}

\begin{CJK}{UTF8}{gkai}   % gkai 楷体 gbsn 宋体 bkai big5編碼的楷體
                          % bsmi big5編碼的明體

  \title{My Note}
  \subtitle{笔记}
  \author{H-X-B \LaTeX{} }
  \institute[HIT]
  {
    哈尔滨工业大学~航天学院\\
    \medskip
    \textit{292554331@qq.com}\\
  }
  \date{\today}

\frame{\titlepage}
% \begin{frame}
%   \titlepage
% \end{frame}

\begin{frame}{目录}
  \tableofcontents
\end{frame}

\section{Emacs}

 \begin{frame}\frametitle{Latex}
   pdf-tool:okular\\
   sudo apt-get install texlive-full\\
   sudo apt-get install texlive-science\\
   sudo apt-get install texlive-lang-CJK\\
 \end{frame}

 \begin{frame}\frametitle{w3m}
   sudo apt-get install w3m\\
   sudo apt-get install w3m-img\\
   sudo apt-get install w3m-el\\
   load-libary w3m\\
 \end{frame}

\section{Cmake}

 \begin{frame}\frametitle{c++ 11}
   How to activate C++ 11 in CMake \\
   http://stackoverflow.com/questions/10851247/how-to-activate-c-11-in-cmake\\
   set(CATKIN\_TOPLEVEL TRUE)\\
   set(CMAKE\_CXX\_STANDARD 11)\\
 \end{frame}

 \begin{frame}\frametitle{OpenCV}
     initModule\_nonfree未定义的引用\\
     solution: http://blog.csdn.net/zyh821351004/article/details/47322823
 \end{frame}

 \begin{frame}\frametitle{G2O}
     In file included from /usr/local/include/g2o/core/optimizable\_graph.h:38:0,
                from /usr/local/include/g2o/core/base\_vertex.h:30,
                from /usr/local/include/g2o/types/slam3d/types\_slam3d.h:21,
                from /home/hxb/rgbdslam/6/src/slamEnd.cpp:9:
     /usr/local/include/g2o/core/hyper\_graph.h:138:15: error:
     ‘unordered\_map’ in namespace ‘std’ does not namea type
   solution:sudo apt-get install ros-indigo-libg2o\\
   sudo cp g2o\_viewer /usr/local/bin/
 \end{frame}

\section{系统}

 \begin{frame}\frametitle{Stardict}
   sudo apt-get install stardict\\
   词库网址:http://download.huzheng.org/\\
   sudo cp -r 解压文件 /usr/share/stardict/dic/
 \end{frame}

 \begin{frame}\frametitle{搜狗拼音}
   网址: http://pinyin.sogou.com/linux/?r=pinyin\\
   im-config 选择 fcitx 然后 reboot\\
   fcitx-config-gtk3\\
   \begin{figure}
     \centering
     \includegraphics[width=10.00cm,height=5.10cm]{test.png}
     %\includegraphics[width=0.5\linewidth]{test.png}
   \end{figure}
 \end{frame}

 \begin{frame}\frametitle{ttyUSB0}
   cd /etc/udev/rules.d/ \\
   touch serial-usb.rules\\
   %\begin{theorem}
    KERNEL=="ttyUSB0",GROUP="uucp",MODE="0666"
   %\end{theorem}
 \end{frame}

 \begin{frame}\frametitle{挂载}
   sudo fdisk -l\\
   sudo mount -t nfts /dev/sda\quotes{n} /mnt/
 \end{frame}

% \begin{frame}\frametitle{change color}
%   cd /usr/share/themes/Ambiance/gtk-3.0\\ %
%   sudo vim gtk-main.css\\
%   base_color #cce8cf\\
% \end{frame}

% \begin{frame}\frametitle{google change color}
%   Google Chrome(谷歌浏览器)修改网页背景颜色的办法(比如修改为护眼的豆沙绿)  \\
%   过客\\
% \end{frame}

\end{CJK}

\end{document}
