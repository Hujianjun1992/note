\documentclass{beamer}
\usetheme{Warsaw}
\usecolortheme{lily}


%\hypersetup{CJKbookmarks=true}
\usepackage{CJKutf8}

\newcommand{\quotes}[1]{"#1"}

\begin{document}

\begin{CJK}{UTF8}{gkai}

    \title{Git}
  \subtitle{笔记}
  \author{H-X-B \LaTeX{} }
  \institute[HIT]
  {
    哈尔滨工业大学~航天学院\\
    \medskip
    \textit{292554331@qq.com}\\
  }
  \date{\today}

  \frame{\titlepage}

  \begin{frame}\frametitle{安装及初始化}
    sudo apt-get install git\\
    git config --global user.name \quotes{My Name}\\
    git config --global user.email \quotes{email@example.com}\\
  \end{frame}

  \begin{frame}\frametitle{创建版本库}
    mkdir learninggit \&\& cd learninggit \&\& git init \\
    git add 文件添加到仓库\\
    git commit -m \quotes{mark} 文件提交到仓库\\
  \end{frame}

  \begin{frame}\frametitle{时光穿梭机}
    git status 查看仓库状态\\
    git diff 查看不同
  \end{frame}

  \begin{frame}\frametitle{版本回退}
    git log 显示从最近到最远的提交日志\\
    git log --pretty=oneline\\
%    git reset --hard HEAD\^ HEAD\^\^ HEAD-100 commit id\\
    git reflog 记录每一次命令\\
  \end{frame}

  \begin{frame}\frametitle{工作区和缓存区}

  \end{frame}

  \begin{frame}\frametitle{撤销修改}
    git checkout -- filename\\
    没有实现 有问题不知道为啥
  \end{frame}

  \begin{frame}\frametitle{删除文件}
    git rm filename\\
    git commit -m \quotes{note}\\
    一键还原 git checkout -- filename 有问题 实现不了
  \end{frame}

  \begin{frame}\frametitle{远程仓库}
步骤网址:http://www.liaoxuefeng.com/wiki/0013739516305929606dd18361248578c67b8067c8c017b000/001374385852170d9c7adf13c30429b9660d0eb689dd43a000\\
    ssh-keygen -t -rsa -C \quotes{youremail@example.com}\\
  \end{frame}

  \begin{frame}\frametitle{添加远程库}
    在本地仓库下:
    git remote add origin git@github.com:GitHub账户名/learngit.git\\
    git push -u origin master 第一次将本体库内容推送到远程 加-u\\
  \end{frame}

  \begin{frame}\frametitle{从远程库克隆}
    创建一个新的仓库  添加readme\\
    git clone git@github.com:GitHub账户名/gitskills.git\\
  \end{frame}

  \begin{frame}\frametitle{分支管理}

  \end{frame}

\end{CJK}

\end{document}